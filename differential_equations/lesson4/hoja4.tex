\documentclass[a4paper,10pt]{article}
% packages
\usepackage[utf8]{inputenc}
\usepackage[spanish]{babel}
\usepackage{geometry}
\usepackage{datetime}  
\geometry{a4paper,total={170mm,257mm},left=20mm,top=20mm,}

\usepackage{amsmath}
\usepackage{amssymb}
\usepackage{physics}
\usepackage{color}
\usepackage{listingsutf8}
\usepackage{subcaption}
\usepackage{siunitx}
\usepackage{graphicx}

\usepackage{blindtext}  
\usepackage{multicol}  


% Title
\title{Solutions of the Exercises of Lesson 4}
\author{Vega Vázquez Ardid}
\date{\today}

\usepackage{natbib}
\usepackage{graphicx}


\begin{document}

\maketitle

\begin{enumerate}
    \item
    For each of the following exercises, sketch all the qualitatively different vector fields that occur as r is varied. Show that a saddle-node bifurcation occurs at a critical value of r, to be determined. Finally, sketch the bifurcation diagram of fixed points x∗ versus r
        \begin{enumerate}
            \item 
                \begin{equation}
                    \Dot{x}= 1 + rx + x^{2}
                \end{equation}
                Analytical solution:
                \begin{equation}
                    \begin{aligned}
                        f(x)= 1 + rx + x^{2}\\
                        f(x)=0\\
                        x= \frac{-r  \pm \sqrt{r^{2}-1}}{2}\\
                        \left.
                        \begin{array}{rcl}
                            r= \pm 2 \xrightarrow{} \text{1 point} \\
                            r= 2 \xrightarrow{} -1\\
                            r= -2 \xrightarrow{} 1\\
                        \end{array}
                        \right\}
                        1)\\
                        2) \hspace{0.2cm}if \hspace{0.2cm}\abs{r}> 2 \xrightarrow{} \text{2 points}\hspace{0.2cm} r >2 , r<-2\\
                        3)\hspace{0.2cm} if \hspace{0.2cm}\abs{r}<2 \xrightarrow{} \emptyset \text{points}\hspace{0.2cm} r<2, r>-2\\ 
                    
                    \end{aligned}
                \end{equation}
               Graphical solution:
               \begin{equation}
                   \begin{aligned}
                       f_{1}(x)= 1+ x^{2}\\
                       f_{2}(x)= -rx
                   \end{aligned}
               \end{equation}
                1)
                \begin{figure}[h]
                    \centering
                    \includegraphics[width=0.40\textwidth]{ej1img11.jpg}
                    
                    \label{fig:mesh1}
                \end{figure}
                \newpage
                \begin{figure}[h]
                    \centering
                    \includegraphics[width=0.40\textwidth]{ej1img12.jpg}
                    
                    \label{fig:mesh1}
                \end{figure}
                2)
                \begin{figure}[h]
                    \centering
                    \includegraphics[width=0.40\textwidth]{ej1img21.jpg}
                    
                    \label{fig:mesh1}
                \end{figure}
                \newpage
                \begin{figure}[h]
                    \centering
                    \includegraphics[width=0.40\textwidth]{ej1img22.jpg}
                    
                    \label{fig:mesh1}
                \end{figure}
                3)
                \begin{figure}[h]
                    \centering
                    \includegraphics[width=0.40\textwidth]{ej1img31.jpg}
                    
                    \label{fig:mesh1}
                \end{figure}
                \newpage
                \begin{figure}[h]
                    \centering
                    \includegraphics[width=0.40\textwidth]{ej1img32.jpg}
                    
                    \label{fig:mesh1}
                \end{figure}
                x vs r
                \begin{figure}[h]
                    \centering
                    \includegraphics[width=0.40\textwidth]{ej1img4.jpg}
                    
                    \label{fig:mesh1}
                \end{figure}
            
            
            \item 
                \begin{equation}
                    \dot{x}= r + x - ln(1+x)
                \end{equation}
                Analytical solution:
                \begin{equation}
                    \begin{aligned}
                        f_{1}(x)= r+x\\
                        f_{2(x)}= ln( 1+x)\\
                        \text{there is only two options, either}\hspace{0.2cm}x> -1, \text{or x= 0}\\
                        \text{In order to see if it increases or decreases we obtain the second derivative}\\
                        f'_{2}(x)=\frac{1}{1+x}> 0 \xrightarrow{} increases
                    \end{aligned}
                \end{equation}
                \begin{figure}[h]
                    \centering
                    \includegraphics[width=0.40\textwidth]{ej1img5.jpg}
                    
                    \label{fig:mesh1}
                \end{figure}
                \newpage
                Graphical solution:
                
                \begin{figure}[h]
                    \centering
                    \includegraphics[width=0.40\textwidth]{ej1img61.jpg}
                    
                    \label{fig:mesh1}
                \end{figure} 
                \newpage
                \begin{figure}[h]
                    \centering
                    \includegraphics[width=0.40\textwidth]{ej1img62.jpg}
                    
                    \label{fig:mesh1}
                \end{figure}
                \begin{figure}[h]
                    \centering
                    \includegraphics[width=0.40\textwidth]{ej1img63.jpg}
                    
                    \label{fig:mesh1}
                \end{figure}
                \newpage
                \begin{figure}[h]
                    \centering
                    \includegraphics[width=0.40\textwidth]{ej1img7.jpg}
                    
                    \label{fig:mesh1}
                \end{figure}
                
        \end{enumerate}
    \item For each of the following exercises, sketch all the qualitatively different vector fields that occur as r is varied. Show that a transcritical bifurcation occurs at a critical value of r, to be determined. Finally,
    sketch the bifurcation diagram of fixed points x∗ vs. r
        \begin{enumerate}
            \item 
                \begin{equation}
                    \begin{aligned}
                    \dot{x}= rx + x^{2}\\
                    rx + x^{2}=0\\
                    x=0\\
                    x=-r
                    \end{aligned}
                    
                \end{equation}
                Graphically:
                \begin{figure}[h]
                    \centering
                    \includegraphics[width=0.40\textwidth]{ej2imga1.jpg}
                    
                    \label{fig:mesh1}
                \end{figure}
                \newpage
                \begin{figure}[h]
                    \centering
                    \includegraphics[width=0.40\textwidth]{ej2imga2.jpg}
                    
                    \label{fig:mesh1}
                \end{figure}
                \begin{figure}[h]
                    \centering
                    \includegraphics[width=0.40\textwidth]{ej2imga3.jpg}
                    
                    \label{fig:mesh1}
                \end{figure}
                \newpage
                \begin{figure}[h]
                    \centering
                    \includegraphics[width=0.40\textwidth]{ej2imga4.jpg}
                    
                    \label{fig:mesh1}
                \end{figure}
            \item
                \begin{equation}
                    \begin{aligned}
                        \dot{x}= x- rx(1-x)\\
                        (1-r)x + rx^{2}\\
                        x= 0\\
                        x= 1 - \frac{1}{r}
                    \end{aligned}
                \end{equation}
               \begin{figure}[h]
                    \centering
                    \includegraphics[width=0.40\textwidth]{ej2imgb1.jpg}
                    \label{fig:mesh1}
                \end{figure}
                \newpage
                \begin{figure}[h]
                    \centering
                    \includegraphics[width=0.40\textwidth]{ej2imgb2.jpg}
                    \label{fig:mesh1}
                \end{figure}
                \begin{figure}[h]
                    \centering
                    \includegraphics[width=0.40\textwidth]{ej2imgb3.jpg}
                    \label{fig:mesh1}
                \end{figure}
                \newpage
                \begin{figure}[h]
                    \centering
                    \includegraphics[width=0.40\textwidth]{ej2imgb4.jpg}
                    \label{fig:mesh1}
                \end{figure}
                
        \end{enumerate}
        
        
    \item In the following exercises, sketch all the qualitatively different vector fields that occur as r is varied. Show that a pitchfork bifurcation occurs at a critical value of r (to be determined) and classify the bifurcation as supercritical or subcritical. Finally, sketch the bifurcation diagram of x∗ vs. r.
        \begin{enumerate}
            \item 
                \begin{equation}
                    \begin{aligned}
                        \dot{x}=rx + 4x^{3}\\
                        x(r+4x^{2})=0\\
                        x=0\\
                        x=\pm\sqrt{\frac{-r}{4}}\\
                        \text{r must be negative in order for three fixed points to exists}
                    \end{aligned}
                \end{equation}
                \begin{figure}[h]
                    \centering
                    \includegraphics[width=0.40\textwidth]{ej3imga1.jpg}
                    \label{fig:mesh1}
                \end{figure}
                \newpage
                \begin{figure}[h]
                    \centering
                    \includegraphics[width=0.40\textwidth]{ej3imga2.jpg}
                    \label{fig:mesh1}
                \end{figure}
                \begin{figure}[h]
                    \centering
                    \includegraphics[width=0.40\textwidth]{ej3imga3.jpg}
                    \label{fig:mesh1}
                \end{figure}
                \newpage
                \begin{figure}[h]
                    \centering
                    \includegraphics[width=0.40\textwidth]{ej3imga4.jpg}
                    \label{fig:mesh1}
                \end{figure}
                \newpage
            \item
                \begin{equation}
                    \begin{aligned}
                        \dot{x}= rx- 4x^{3}\\
                        \text{opposite case to a)}\\
                        x=0\\
                        r-4x^{2}=0\\
                        x=\pm\sqrt{\frac{r}{4}}\\
                        \text{r must be positive in order for three fixed points to exists}
                    \end{aligned}
                \end{equation}
                
                \begin{figure}[h]
                    \centering
                    \includegraphics[width=0.40\textwidth]{ej3imgb1.jpg}
                    \label{fig:mesh1}
                \end{figure}
                \begin{figure}[h]
                    \centering
                    \includegraphics[width=0.40\textwidth]{ej3imgb2.jpg}
                    \label{fig:mesh1}
                \end{figure}
               \newpage
                \begin{figure}[h]
                    \centering
                    \includegraphics[width=0.40\textwidth]{ej3imgb3.jpg}
                    \label{fig:mesh1}
                \end{figure}
                \begin{figure}[h]
                    \centering
                    \includegraphics[width=0.40\textwidth]{ej3imgb4.jpg}
                    \label{fig:mesh1}
                \end{figure}
        \end{enumerate}
        
    \item The next exercises are designed to test your ability to distinguish among the various types of bifurcations it’s easy to confuse them! In each case, find the values of r at which bifurcations occur, and classify those as saddle-node, transcritical, supercritical pitchfork, or subcritical pitchfork. Finally, sketch the bifurcation diagram of fixed points x∗ vs. r.
        \begin{enumerate}
            \item 
                \begin{equation}
                    \begin{aligned}
                        \dot{x}= r- 3x^{2}\\
                        r- 3x^{2}=0\\
                        x=\pm \sqrt{\frac{r}{3}}\\
    
                    \end{aligned}
                \end{equation}
                \newpage
                Graphically:
                \begin{figure}[h]
                    \centering
                    \includegraphics[width=0.40\textwidth]{ej4imga1.jpg}
                    \label{fig:mesh1}
                \end{figure}
                \begin{figure}[h]
                    \centering
                    \includegraphics[width=0.40\textwidth]{ej4imga2.jpg}
                    \label{fig:mesh1}
                \end{figure}
                \newpage
                \begin{figure}[h]
                    \centering
                    \includegraphics[width=0.40\textwidth]{ej4imga3.jpg}
                    \label{fig:mesh1}
                \end{figure}
                \begin{figure}[h]
                    \centering
                    \includegraphics[width=0.40\textwidth]{ej4imga4.jpg}
                    \label{fig:mesh1}
                \end{figure}
                
            \item
                \begin{equation}
                    \begin{aligned}
                        \dot{x}= 5- re^{-x^{2}}\\
                        x= \pm\sqrt{-ln\frac{5}{r}}\\
                        
                    \end{aligned}
                \end{equation}
                \newpage
                Graphically:
                
                \begin{figure}[h]
                    \centering
                    \includegraphics[width=0.40\textwidth]{ej4imgb1.jpg}
                    \label{fig:mesh1}
                \end{figure}
                
                \begin{figure}[h]
                    \centering
                    \includegraphics[width=0.40\textwidth]{ej4imgb2.jpg}
                    \label{fig:mesh1}
                \end{figure}
                \newpage
                \begin{figure}[h]
                    \centering
                    \includegraphics[width=0.40\textwidth]{ej4imgb3.jpg}
                    \label{fig:mesh1}
                \end{figure}
                
                \begin{figure}[h]
                    \centering
                    \includegraphics[width=0.40\textwidth]{ej4imgb4.jpg}
                    \label{fig:mesh1}
                \end{figure}
                \newpage
                
            \item
                \begin{equation}
                    \begin{aligned}
                        \dot{x}= x + tanh(rx)\\
                        tanh(rx)= -x\\
                    \end{aligned}
                
                \end{equation}
                \begin{figure}[h]
                    \centering
                    \includegraphics[width=0.40\textwidth]{ej4imgc1.jpg}
                    \label{fig:mesh1}
                \end{figure}
                \newpage
                Graphically:
                \begin{figure}[h]
                    \centering
                    \includegraphics[width=0.40\textwidth]{ej4imgc2.jpg}
                    \label{fig:mesh1}
                \end{figure}
                \begin{figure}[h]
                    \centering
                    \includegraphics[width=0.40\textwidth]{ej4imgc3.jpg}
                    \label{fig:mesh1}
                \end{figure}
                
                \begin{equation}
                    \begin{aligned}
                        if \hspace{0.2cm}r< 0\\
                        f(x)= x+ tanh(x)\\
                        t=-r\\
                        t>0\\
                        x+ tanh(-tx)= x-tanh(tx9\\
                        if \hspace{0.2cm}r= -1\\
                        tanh(-x)=-x\\
                        -tanh(x)= -x\\
                        x=0 \hspace{0.2cm}\text{fixed point}\\
                        if \hspace{0.2cm}r<-1\\
                        tanh(rx)= -x\\
                        \text{ 3 fixed points}
                    \end{aligned}
                \end{equation}
                \newpage
                \begin{figure}[h]
                    \centering
                    \includegraphics[width=0.40\textwidth]{ej4imgc4.jpg}
                    \label{fig:mesh1}
                \end{figure} 
                \begin{equation}
                    \begin{aligned}
                        -1<r<0\\
                        x=0, unstable\\
                        -tanh(x)= -x\\
                        -tanh(x)+x=0
                    \end{aligned}
                \end{equation}
                \newpage
                \begin{figure}[h]
                    \centering
                    \includegraphics[width=0.40\textwidth]{ej4imgc5.jpg}
                    \label{fig:mesh1}
                \end{figure}
                \begin{equation}
                    \begin{aligned}
                        r<-1\\
                        tanh(rx) + x=0
                    \end{aligned}
                \end{equation}
                \begin{figure}[h]
                    \centering
                    \includegraphics[width=0.40\textwidth]{ej4imgc6.jpg}
                    \caption{pitchfork x=-1}
                    \label{fig:mesh1}
                \end{figure}
               
        \end{enumerate}
        
    \newpage    
    \item Calculate rc, where rc is defined by the condition that V has three equally deep wells, i.e., the values of V at the three local minima are equal. (Note: In equilibrium statistical mechanics, one says that a first-order phase transition occurs at r = rc. For this value of r, there is equal probability of finding the system in the state corresponding to any of the three minima. The freezing of water into ice is the most familiar example of a first-order phase transition.) 
            \begin{equation}
                \begin{aligned}
                    f(x)= \frac{-dv}{dx}\\
                    \frac{dv}{dx}= -rx -x^{3}-x^{5}\\
                     -rx -x^{3}-x^{5}=0\\
                     x_{1}=0\\
                     t^{2}=x^{4}\\
                     t=x^{2}\\
                     -t^{2} + t +r=0
                     t=\frac{-1\pm \sqrt{1+ 4r}}{-2}= \frac{-1\pm\sqrt{1+ 4r}}{2}\\
                     x^{2}= \frac{1\pm\sqrt{1+4r}}{2}\\
                     x_{2}= \sqrt{\frac{1+\sqrt{1+4r}}{2}}\\
                     x_{3}= -\sqrt{\frac{1+\sqrt{1+4r}}{2}}\\
                     x_{4}=\sqrt{\frac{1-\sqrt{1+4r}}{2}}\\
                     x_{5}=-\sqrt{\frac{1-\sqrt{1+4r}}{2}}\\
                     \text{calculating the local minimum}\\
                     \frac{d^{2}}{dx}= -r -3x^{2}+5x^{4}>0\\
                     \frac{d^{2}}{dx}(x_{1})=-r>0 if \hspace{0.2cm}r<0\\
                     \frac{d^{2}}{dx}(x_{2})= -r -\frac{3}{2}(1+\sqrt{1+4r})+5((\sqrt{\frac{1+\sqrt{1+4r}}{2}})^{2})^{2}=\\
                     =-r +\frac{3}{2}(1+\sqrt{1+4r}) + 5(1+2r+\sqrt{1+4r})>0\\
                     4r>-1\\
                     r> \frac{-1}{4}\\
                     x_{2}=x_{3}\\
                     \frac{d^{2}}{dx}(x_4)=-r +\frac{3}{2}(1+\sqrt{1+4r}) + 5( 1 + 2r - \sqrt{1+4r})>0 \\
                     r>0\\
                     x_{4}=x_{5}\\
                     x_{1}=x_{2}=x_{3}\\
                     v=\int -f(x)dx= -r\frac{x^{2}}{2} - \frac{x^{4}}{4} + \frac{x^{6}}{6}\\
                     \frac{-r}{2}(\sqrt{1+ \sqrt{1+4r}})^{2} - \frac{1}{4}(\sqrt{1+\sqrt{1+4r}})^{4} + \frac{1}{6}(\sqrt{1+\sqrt{1+4r}})^{6}=\\
                     = \frac{-r}{2}(-\sqrt{1+ \sqrt{1+4r}})^{2} - \frac{1}{4}(-\sqrt{1+\sqrt{1+4r}})^{4} +\frac{1}{6}(-\sqrt{1+\sqrt{1+4r}})^{6}\\
                     r= \frac{-3}{16}
                \end{aligned}
            \end{equation}
            \newpage
            \begin{figure}[h]
                \centering
                \includegraphics[width=0.40\textwidth]{ej5img1.jpg}
                \caption{pitchfork x=-1}
                \label{fig:mesh1}
            \end{figure}
    \newpage         
    \item (Nondimensionalizing the logistic equation) Consider the logistic equation N˙ = rN(1−N/K), with initial condition N(0) = N0. 
        \begin{enumerate}
            \item This system has three dimensional parameters r, K, and N0. Find the dimensions of each of these parameters.
                \begin{equation}
                    \begin{aligned}
                        \frac{dN}{dt}= rN(1-\frac{N}{K})\\
                        \frac{dN}{N}=\frac{1}{t}dt(1-\frac{N}{K})\\
                        k\longrightarrow population\\
                        N_{o}\longrightarrow population\\
                        [r]= \longrightarrow\frac{1}{t} \text{so units workout}

                    \end{aligned}
                \end{equation}
            \item Show that the system can be rewritten in the dimensionless form
                \begin{equation}
                    \begin{aligned}
                        \frac{dx}{dt}=x(1-x)\\
                        x(0)= x_{0}\\
                        x=\frac{N}{K}\\
                        N=xK\\
                        \dot{x}=x(1-x)\\
                        K\dot{x}=rN(1-x)\\
                        k\dot{x}=rxK(1-x)\\
                        \dot{x}=rx(1-x)\\
                        \frac{dx}{dt}=rx(1-x)\\
                        r \frac{dx}{dt}=rx(1-x)\\
                        \frac{dx}{dt}=x(1-x)\\
                        x(0)=x_{0}\\
                        \boxed{\frac{dx}{dt}=x(1-x)}\\
                        \boxed{\dot{x}=x(1-x))}

                    \end{aligned}
                \end{equation}
            \item  Find a different nondimensionalization in terms of variables u and τ, where u is chosen such that the initial condition is always u0 = 1
                \begin{equation}
                    \begin{aligned}
                        \dot{N}=rN(1-\frac{N}{K})\\
                        N(0)=N_{0}\\
                        N=uN_{0}
                        u=\frac{N}{N_{0}}\\
                        u(0)=1\\
                        \dot{u}=\frac{dN}{N_{0}}\\
                        \boxed{dN=\dot{u}=N_{0}}\\
                        dN0=rN_{0}u(1+\frac{N_{0}}{K}u)\\
                        \dot{u}N_{0}=rN_{0}u(1-\frac{N_{0}}{K}u)\\
                        \dot{u}=ru(1-\frac{N_{0}}{K}u)\\
                        \frac{1}{r}\dot{u}=u(1-\frac{N_{0}}{K}u)\\
                        \frac{1}{r}\frac{du}{dt}=u(1-\frac{N_{0}}{K}u)\\
                        \tau=rt
                        d\tau=rdt\\
                        dt=\frac{d\tau}{r}\\
                        \frac{du}{d\tau}=u(1-\frac{N_{0}}{K}u)\\
                        \boxed{\frac{du}{d\tau}=u(1-ku), u(0)=1}

                    \end{aligned}
                \end{equation}
            
        \end{enumerate}
       
    
      
\end{enumerate}
\end{document}