\documentclass[a4paper,10pt]{article}
% packages
\usepackage[utf8]{inputenc}
\usepackage[spanish]{babel}
\usepackage{geometry}
\usepackage{datetime}  
\geometry{a4paper,total={170mm,257mm},left=20mm,top=20mm,}

\usepackage{amsmath}
\usepackage{amssymb}
\usepackage{physics}
\usepackage{color}
\usepackage{listingsutf8}
\usepackage{subcaption}
\usepackage{siunitx}
\usepackage{graphicx}

\usepackage{blindtext}  
\usepackage{multicol}  


% Title
\title{Solutions of the Exercises of Lesson 5}
\author{Vega Vázquez Ardid}
\date{\today}

\usepackage{natbib}
\usepackage{graphicx}


\begin{document}

\maketitle

\begin{enumerate}
    \item 1. Classify the stability of the following systems in: Liapunov stable, asymptotically stable, or none of the above.
        \begin{enumerate}
            \item
                \begin{equation}
                    \begin{aligned}
                        \Dot{x}=y\\
                        \Dot{y}=-4x\\
                    \end{aligned}
                \end{equation}
                \begin{equation}
                    \begin{bmatrix}
                        \dot{x}\\
                        \dot{y}
                    \end{bmatrix}
                    =
                    \begin{bmatrix}
                        0 & 1\\
                        -4 & 0
                    \end{bmatrix}
                    \begin{bmatrix}
                        x\\
                        y
                    \end{bmatrix}
                \end{equation}
                \begin{equation}
                    \left.
                    \Delta = 4 > 0 \atop
                    \tau=0
                    \right\}{centers}
                \end{equation}
                \begin{equation}
                    \begin{aligned}
                        p(\lambda)= \lamda^{2} + 4=0\\
                        \lambda= \pm2i\\
                        x(t)= c1e^{\lambda_{1}t}\vec{v_{1}} + c2e^{\lambda_{2}t}\vec{v_{2}}\\
                        \lambda_{1}= \alpha_{1} + \beta_{1}i\\
                        e^{\lambda_{1}t}= e^{(\alpha_{1} + \beta_{1}i)t}= e^{\alpha_{1}t}e^{\beta_{1}it}= e^{\alpha_{1}t}(cos(\beta_{1}t) + isen(\beta_{2}t))\\
                        y= \dot{x}\\
                        \dot{y}= \"x\\
                        \dot{y}= -4x\\
                        \"x= -4x\\
                        \"x+4x= 0\\
                        \text{homogeneous EDO,second order}\\
                        x(t)=c1cos(2t) + c2sen(2t)\\
                        y(t)=-c1sent(2t) + 2c2cos(2t)\\
                        (x_{0}, y_{0}), t=0\\
                        x_{0}=c1\\
                        y_{0}= 2c2, c2= \frac{y_{0}}{2}\\
                        x(t)= x_{0}cos(2t) + \frac{y_{0}}{2}sen(2t)\\
                        y(t)= -x_{0}sen(2t) + y_{0}cos(2t)\\
                        \text{if  t} \to \infty\\
                        x(t), y(t) \leq 1\\
                        Because:\\
                        \abs{sent(2t)}\leq 1\\
                        \abs{cos(2t)}\leq 1\\
                        \textbf{Liapunov stable}
                            
                    \end{aligned}
                \end{equation}
                
                
            \item 
                \begin{equation}
                    \begin{aligned}
                        \dot{x}= 2y\\
                        \dot{y}= x
                    \end{aligned}
                \end{equation}
                \begin{equation}
                    \begin{bmatrix}
                        \dot{x}\\
                        \dot{y}
                    \end{bmatrix}
                    =
                    \begin{bmatrix}
                        0 & 2\\
                        0 & 1
                    \end{bmatrix}
                    \begin{bmatrix}
                        x\\
                        y
                    \end{bmatrix}
                    
                \end{equation}
                \begin{equation}
                    \left.
                    \Delta =  0 \atop
                    \tau=1
                    \right\}{\text{line of unstable fixed points}}
                \end{equation}
                \begin{equation}
                    \begin{aligned}
                        p(\lambda)= \lambda^{2}-\lambda=0\\
                        \lambda_{1}=0\\
                        \lambda_{2}=1\\
                        x(t) = c1 + c2e^{t}\vec{v2}\\
                        Eigenvectors:\\
                        \abs{\Delta - \lambdaI_{2}}v=0\\
                        \lambda=0\\
                        \begin{bmatrix}
                            0 & 2\\
                            0 &1 
                        \end{bmatrix}
                        \begin{bmatrix}
                            v1\\
                            v2
                        \end{bmatrix}
                        = 
                        \begin{bmatrix}
                            0\\
                            0
                        \end{bmatrix}
                        2v2 + v2=0\\
                        3v2=0\\
                        v(1,0)
                        \lambda=1
                        \begin{bmatrix}
                            -1 & 2\\
                            0 & 0 
                        \end{bmatrix}
                        \begin{bmatrix}
                            v1\\
                            v2
                        \end{bmatrix}
                        = 
                        \begin{bmatrix}
                            0\\
                            0
                        \end{bmatrix}
                        -v1 +v2=0\\
                        v2=v1\\
                        v2(1,1)
                    \end{aligned}
                \end{equation}
            \item
                \begin{equation}
                    \begin{aligned}
                        x=x_{0}\\
                        \dot{y}=x_{0}\\
                        y =x_{0}t + y\\
                       c\text{if t}\to \infty\\
                        \text{uniform motion}\\
                        \textbf{trajectories tend to infinite, unstable}
                    
                    \end{aligned}
                \end{equation}
            \item
                \begin{equation}
                    \begin{aligned}
                        \dot{x}= 0\\
                        \dot{y}= -y\\
                    \end{aligned}
                \end{equation}
                \begin{equation}
                    \begin{bmatrix}
                        \dot{x}\\
                        \dot{y}
                    \end{bmatrix}
                    =
                    \begin{bmatrix}
                        0 & 0\\
                        0 & -1
                    \end{bmatrix}
                    \begin{bmatrix}
                        x\\
                        y
                    \end{bmatrix}
                \end{equation}
                \begin{equation}
                    \left.
                    \Delta = 0 \atop
                    \tau=-1 
                    \right\}{\text{line of stable fixed points}}
                \end{equation}
                \begin{equation}
                    \begin{aligned}
                        p(\lambda)= \lambda^{2} + \lambda=0\\
                        \lambda_{1}= 0\\
                        \lambda_{2}= -1\\
                        x(t)= c1\vec{v1} + c2e^{-t}\vec{v2}\\
                        Eigenvectors:\\
                        \lambda=0\\
                        \begin{bmatrix}
                            0 & 0\\
                            0 &-1 
                        \end{bmatrix}
                        \begin{bmatrix}
                            v1\\
                            v2
                        \end{bmatrix}
                        = 
                        \begin{bmatrix}
                            0\\
                            0
                        \end{bmatrix}
                        v(1,0)\\
                        \lambda= -1\\
                        \begin{bmatrix}
                            1 & 0\\
                            0 &0 
                        \end{bmatrix}
                        \begin{bmatrix}
                            v1\\
                            v2
                        \end{bmatrix}
                        = 
                        \begin{bmatrix}
                            0\\
                            0
                        \end{bmatrix}
                        v(0,1)\\
                        \begin{bmatrix}
                            x_{0}\\
                            y_{0}
                        \end{bmatrix}
                        =
                        \begin{bmatrix}
                            1\\
                            0
                        \end{bmatrix}c1
                        \begin{bmatrix}
                            0\\
                            1
                        \end{bmatrix}c2\\
                        x_{0}= c1\\
                        y_{0}=c2\\
                        x(t)= x_{0}\\
                        y(t)= y_{0}e^{-t}\\
                        \text{if  t} \to \infty\\
                        y(t) \to 0 \\
                        \textbf{Liapunov stable}
                    \end{aligned}
                \end{equation}
                
            \item 
                \begin{equation}
                    \begin{aligned}
                        \left.
                        \dot{x}=-x \atop
                        \dot{y}= -5y
                        \right\}{\text{decoupled system}}\\
                        \text{separable variables}\\
                        \frac{dx}{dt}= -x\\
                        \frac{-dx}{x}= dt\\
                        -ln(x)= t +c\\
                        ln(x)= -t +c\\
                        x(t)= e^{-t}c\\
                        t=0\\
                        x(0)=x_{0}\\
                        \boxed{x(t)= x_{0}e^{-t}}\\
                        \dot{y}=-5y\\
                        \frac{dy}{y}= -5dt\\
                        ln(y)= -5t +c\\
                        t=0\\
                        y(0)=y_{0}\\
                        y=e^{-5t}c\\
                        \boxed{y=e^{-5t}y_{0}}\\
                        \text{if  t} \to \infty\\
                        (x_{0},y_{0})=(0,0)\\
                        \textbf{Asymptotically stable}
                    
                    \end{aligned}
                \end{equation}
            \item 
                \begin{equation}
                    \begin{aligned}
                        \left.
                        \dot{x}=x \atop
                        \dot{y}= y
                        \right\}{\text{decoupled system}}\\
                        \text{separable variables}\\
                        \frac{dx}{dt}= x\\
                        ln(x)= t+c\\
                        x(t)= e^{t}c\\
                        t=0\\
                        x(0)=x_{0}\\
                        x(t)=x_{0}e^{t}\\
                        \frac{dy}{y}= y\\
                        Ln(y)= t +c\\
                        y= e^{t}c\\
                        t=0\\
                        y(0)= y_{0}\\
                        y(t)=y_{0}e^{t}\\
                        \text{if t}\to \infty\\
                        \text{trajectories go away from origin}\\
                        \textbf{unstable}
                    \end{aligned}
                \end{equation}
        \end{enumerate}
    \item Consider the system
        \begin{enumerate}
            \item Write the system as x˙ = Ax.
                \begin{equation}
                    \begin{aligned}
                        \dot{x}= 4x-y\\
                        \dot{y}= 2x +y
                    \end{aligned}
                \end{equation}
                \begin{equation}
                    \begin{bmatrix}
                        \dot{x}\\
                        \dot{y}
                    \end{bmatrix}
                    =
                    \begin{bmatrix}
                        4 & -1\\
                        2 & 1
                    \end{bmatrix}
                    \begin{bmatrix}
                        x\\
                        y
                    \end{bmatrix}
                \end{equation}
                \begin{equation}
                    \begin{aligned}
                        \left.
                        \Delta= 6\\
                        \tau=5
                        \right\}\hspace{0.2cm}{source}\\
                        p(\lambda)= \lambda^{2}-5\lambda+6=0\\
                        \lambda_{1}= 3\\
                        \lambda_{2}= 2\\
                        Eigenvectors:\\
                        \lambda_{1}= 3\\
                        \begin{bmatrix}
                            1 &-1\\
                            2 -2
                        \end{bmatrix}
                        \begin{bmatrix}
                            v1\\
                            v2
                        \end{bmatrix}
                        =
                        \begin{bmatrix}
                            0\\
                            0
                        \end{bmatrix}\\
                        v1-v2=0\\
                        v1=v2\\
                        v1(1,1)\\
                        \lambda_{2}= 2\\
                        \begin{bmatrix}
                            2 &-1\\
                            2 -1
                        \end{bmatrix}
                        \begin{bmatrix}
                            v1\\
                            v2
                        \end{bmatrix}
                        =
                        \begin{bmatrix}
                            0\\
                            0
                        \end{bmatrix}\\
                        2v1-v2=0\\
                        2v1=v2\\
                        v2(1,2)
                    \end{aligned}
                \end{equation}
                
            \item Find the general solution of the system.
                \begin{equation}
                    \begin{aligned}
                        x(t)= c1e^{3t}\vec{v1} + c2e^{2t}\vec{v2}\\
                        x(t)=
                        \begin{bmatrix}
                            1\\
                            1
                        \end{bmatrix}c1e^{3t}
                        +
                        \begin{bmatrix}
                            1\\
                            2
                        \end{bmatrix}c2e^{2t}
                    \end{aligned}
                \end{equation}
                
            \item Classify the fixed point at the origin
                \begin{equation}
                    \begin{aligned}
                        \Delta >0\\
                        \tau>0\\
                        \Delta <\frac{1}{4}\tau^{2}\\
                        \text{source, unstable}
                    \end{aligned}
                \end{equation}
               
                \begin{figure}[h]
                    \centering
                    \includegraphics[width=0.40\textwidth]{ej2imgc.jpg}
                    \caption{classification in axes trace and determinant}
                    \label{fig:mesh1}
                \end{figure}
            \item Solve the system subject to the initial condition (x0, y0) = (3, 4).
                \begin{equation}
                    \begin{bmatrix}
                        3\\
                        4
                    \end{bmatrix}
                        =
                    \begin{bmatrix}
                        1\\
                        1
                    \end{bmatrix}c1
                        +
                    \begin{bmatrix}
                        1\\
                        2
                    \end{bmatrix}c2
                \end{equation}
                \begin{equation}
                    \begin{aligned}
                        \left.
                        3= c1 + c2\atop
                        4= c1+ c2
                        \right\}\\
                        c1= 2\\
                        c2= 1\\
                        \left.
                        x(t)= 2e^{3t} + e^{2t}\atop
                        y(t)=
                        \right\}
                    \end{aligned}
                \end{equation}
        \end{enumerate}
    \item Plot the phase portrait and classify the fixed point of the following linear systems. If the eigenvectors are real, indicate them in your sketch.
        \begin{enumerate}
            \item
                \begin{equation}
                    \begin{aligned}
                        \dot{x}=y\\
                        \dot{y}=-2x-3y
                    \end{aligned}
                \end{equation}
                \begin{equation}
                    \begin{bmatrix}
                        \dot{x}\\
                        \dot{y}
                    \end{bmatrix}
                    =
                    \begin{bmatrix}
                        0 &1\\
                        -2 &-3
                    \end{bmatrix}
                    \begin{bmatrix}
                        x\\
                        y
                    \end{bmatrix}
                \end{equation}
                \begin{equation}
                    \begin{aligned}
                        p(\lambda)= \lambda^{2}+3\lambda +2=0\\
                        \lambda_{1}=-1\\
                        \lambda_{2}= -2\\
                        Eigenvectors:\\
                        \lambda=-1\\
                        \begin{bmatrix}
                            1 &1\\
                            -1&-3
                        \end{bmatrix}
                        \begin{bmatrix}
                            v1\\
                            v2
                        \end{bmatrix}
                        =
                        \begin{bmatrix}
                            0\\
                            0
                        \end{bmatrix}\\
                        v2=-v1\\
                        v1(1,-1)\\
                        \lambda= -2\\
                        \begin{bmatrix}
                            2 &1\\
                            -2&-1
                        \end{bmatrix}
                        \begin{bmatrix}
                            v1\\
                            v2
                        \end{bmatrix}
                        =
                        \begin{bmatrix}
                            0\\
                            0
                        \end{bmatrix}\\
                        2v1=-v2\\
                        v2=(2,-1)\\
                        \Delta>0\\
                        \tau<0\\
                        \Delta <\frac{1}{4}\tau^{2}\\
                        \textbf{sink, stable}
                    \end{aligned}
                \end{equation}
                \newpage
                \begin{figure}[h]
                    \centering
                    \includegraphics[width=0.40\textwidth]{ej3imga.jpg}
                    \label{fig:mesh1}
                \end{figure}
            \item 
                \begin{equation}
                    \begin{aligned}
                        \dot{x}= 3x-4y\\
                        \dot{y}= x- y
                    \end{aligned}
                \end{equation}
                \begin{equation}
                    \begin{bmatrix}
                        \dot{x}\\
                        \dot{y}    
                    \end{bmatrix}
                    =
                    \begin{bmatrix}
                        3 &-4\\
                        1 &-1
                    \end{bmatrix}
                    \begin{bmatrix}
                        x\\
                        y
                    \end{bmatrix}
                \end{equation}
                \begin{equation}
                    \begin{aligned}
                        \Delta=1\\
                        \tau= 2\\
                        p(\lambda)= \lambda^{2}-2\lambda +1=0\\
                        Eigenvectors:\\
                        \lambda=1\\
                        \begin{bmatrix}
                            2&-4\\
                            1 &-2
                        \end{bmatrix}
                        \begin{bmatrix}
                            v1\\
                            v2
                        \end{bmatrix}
                        =
                        \begin{bmatrix}
                            0\\
                            0
                        \end{bmatrix}\\
                        2v1-4v2=0\\
                        v1(2,4)\\
                        \Delta>0\\
                        \tau>0\\
                        \Delta =\frac{1}{4}\tau^{2}\\
                        \text{one eigenvector}\\
                        \textbf{Degenerate source}
                    \end{aligned}
                \end{equation}
                \newpage
                \begin{figure}[h]
                    \centering
                    \includegraphics[width=0.40\textwidth]{ej3imgb.jpg}
                    \label{fig:mesh1}
                \end{figure}
                
            \item 
                \begin{equation}
                    \begin{aligned}
                        \dot{x}= 5x +2y\\
                        \dot{y}=-17x-5y
                    \end{aligned}
                \end{equation}
                \begin{equation}
                    \begin{bmatrix}
                        \dot{x}\\
                        \dot{y}    
                    \end{bmatrix}
                    =
                    \begin{bmatrix}
                        5 & 2\\
                        -17 &-5
                    \end{bmatrix}
                    \begin{bmatrix}
                        x\\
                        y
                    \end{bmatrix}
                \end{equation}
                \begin{equation}
                    \begin{aligned}
                        \Delta=9\\
                        \tau= 0\\
                        p(\lambda)=\lambda^{2} +9\\
                        \lambda_{1}= 3i\\
                        \lambda_{2}=-3i\\
                        Eigenvectors:\\
                        \lambda= 3i\\
                        \begin{bmatrix}
                            5-3i& 2\\
                            -17& -5-3i
                        \end{bmatrix}
                        \begin{bmatrix}
                            v1\\
                            v2
                        \end{bmatrix}
                        =
                        \begin{bmatrix}
                            0\\
                            0
                        \end{bmatrix}\\
                        v1(5-3i, -2)\\
                        \lambda=-3i\\
                        \begin{bmatrix}
                            5+3i& 2\\
                            -17& -5+3i
                        \end{bmatrix}
                        \begin{bmatrix}
                            v1\\
                            v2
                        \end{bmatrix}
                        =
                        \begin{bmatrix}
                            0\\
                            0
                        \end{bmatrix}\\
                        v2(5+3i,-2)\\
                        \Delta > 0\\
                        \tau =0\\
                        \textbf{centers}
                    \end{aligned}
                \end{equation}
                \newpage
                \begin{figure}[h]
                    \centering
                    \includegraphics[width=0.40\textwidth]{ej3imgc.jpg}
                    \label{fig:mesh1}
                \end{figure}
            \item 
                \begin{equation}
                    \begin{aligned}
                        \dot{x}=4x-3y\\
                        \dot{y}=8x -6y
                    \end{aligned}
                \end{equation}
                \begin{equation}
                    \begin{bmatrix}
                        \dot{x}
                        \dot{y}
                    \end{bmatrix}
                    =
                    \begin{bmatrix}
                        4 & -3\\
                        8 & -6
                    \end{bmatrix}
                    \begin{bmatrix}
                        x\\
                        y
                    \end{bmatrix}
                \end{equation}
                \begin{equation}
                    \begin{aligned}
                        \Delta=0\\
                        \tau=-2\\
                        p(\lambda)= \lambda^{2}+2\lambda\\
                        \lambda_{1}= 0\\
                        \lambda_{2}= -2\\
                        Eigenvectors:\\
                        \lambda= 0\\
                        \begin{bmatrix}
                            4 & -3\\
                            8 & -6
                        \end{bmatrix}
                        \begin{bmatrix}
                            v1\\
                            v2
                        \end{bmatrix}
                        =
                        \begin{bmatrix}
                            0\\
                            0
                        \end{bmatrix}\\
                        v1(4,3)\\
                        \lambda= -2\\
                        \begin{bmatrix}
                            6 & -3\\
                            8&-4
                        \end{bmatrix}
                        \begin{bmatrix}
                            v1\\
                            v2
                        \end{bmatrix}
                        =
                        \begin{bmatrix}
                            0\\
                            0
                        \end{bmatrix}\\
                        v2(2,1)\\
                        \Delta=0\\
                        \tau<0\\
                        \textbf{line of stable fixed points}
                    \end{aligned}
                \end{equation}
                \newpage
                \begin{figure}[h]
                    \centering
                    \includegraphics[width=0.40\textwidth]{ej3imgd.jpg}
                    \label{fig:mesh1}
                \end{figure}
        \end{enumerate}
        
    \item . Show that any matrix of the form A = 0 has only a one-dimensional eigenspace corresponding to the eigenvalue λ. Then solve the system x˙ = Ax and sketch the phase portrait.
        \begin{equation}
            \begin{bmatrix}
                a & b\\
                c & d
            \end{bmatrix}
        \end{equation}
        c=0\\
        \begin{equation}
            \begin{aligned}
                \dot{x}=\lambda x + by\\
                \dot{y}=\lambda y\\
                \frac{dy}{dt}=\lambda y\\
                ln(y)= \lambda t +c \\
                y(t)= ce^{\lambda t}\\
                \dot{x}=\lambda x + by\\
                \dot{x}=\lambda x + bce^{\lambda t}\\
                \dot{x}-\lambda x = bce^{\lambda t}\\
                e^{-\lambda t}(\dot{x}-\lambda x)=bc\\
                e^{-\lambda t}\dot{x}-e^{-\lambda t}\lambda x=bc\\
                d(xe^{-\lambda t})=e^{-\lambda t}\dot{x}-e^{-\lambda t}\lambda x\\
                d(xe^{-\lambda t})=bc\\
                xe^{-\lambda t}=\int bcdt\\
                x=e^{\lambda t}(bct+k)\\
                y= ce^{\lambda t}
                Eigenvectors:\\
                \begin{bmatrix}
                    \lambda-\delta & b\\
                    0 & \lambda-\delta
                \end{bmatrix}
                =0\\
                (\lambda-\delta)^{2}=0\\
                \lambda=\delta (double)\\
                \begin{bmatrix}
                    0& b\\
                    0& 0
                \end{bmatrix}
                \begin{bmatrix}
                    v1\\
                    v2
                \end{bmatrix}
                =
                \begin{bmatrix}
                    0\\
                    0
                \end{bmatrix}\\
                bv2=0\\
                v(1,0)\\
                \textbf{one dimensional eigenspace vector}
            \end{aligned}
        \end{equation}
        \newpage
            \begin{figure}[h]
                \centering
                \includegraphics[width=0.40\textwidth]{ej4img.jpg}
                \label{fig:mesh1}
            \end{figure}
            
    \item . (Damped harmonic oscillator) The motion of a damped harmonic oscillator is described by mx¨+bx˙ +kx  0, where b > 0 is the damping constant.
        \begin{enumerate}
            \item  Rewrite the equation as a two-dimensional linear system.Rewrite the equation as a two-dimensional linear system.
                \begin{equation}
                    \begin{aligned}
                        \"x=\frac{-(b\dot{x}+kx)}{m}\\
                        \dot{y}=\"x=\frac{-(b\dot{x}+kx)}{m}\\
                        \dot{x}=y\\
                        \frac{-(b\dot{x}+kx)}{m}= \frac{-by}{m}-\frac{kx}{m}\\
                        \begin{bmatrix}
                            \dot{x}\\
                            \dot{y}
                        \end{bmatrix}
                        =
                        \begin{bmatrix}
                            o & y\\
                            \frac{-k}{m}& \frac{-b}{m}
                        \end{bmatrix}
                        \begin{bmatrix}
                            x\\
                            y
                        \end{bmatrix}
                        
                    \end{aligned}
                \end{equation}
            
            
            \item Classify the fixed point at the origin and sketch the phase portrait. Be sure to show all the different cases that can occur, depending on the relative sizes of the parameters.
                \begin{equation}
                    \begin{aligned}
                        \Delta= \frac{k}{m}y\\
                        \tau= \frac{-b}{m}\\
                        p(\lambda)=\lambda^{2}+ \frac{b}{m}\lambda +\frac{k}{m}y=0\\
                        \lambda= \frac{\frac{-b}{m}\pm\sqrt{(\frac{b}{m})^{2}-4\frac{k}{m}y}}{2}\\
                        \lambda= \frac{-b\pm\sqrt{b^{2}-4ky}}{2m}\\
                    \end{aligned}
                \end{equation}
                \begin{equation}
                    \begin{aligned}
                        if\hspace{0.2cm} b^{2}-4ky > 0\\
                        \textbf{stable node, source}
                    \end{aligned}
                \end{equation}
                \newpage
                \begin{figure}[h]
                    \centering
                    \includegraphics[width=0.40\textwidth]{ej5imgb1.jpg}
                    \label{fig:mesh1}
                \end{figure}
                \begin{equation}
                    \begin{aligned}
                        if\hspace{0.2cm} b^{2}-4ky < 0\\
                        \textbf{spiral source}
                    \end{aligned}
                \end{equation}
                \newpage
                \begin{figure}[h]
                    \centering
                    \includegraphics[width=0.40\textwidth]{ej5imgb2.jpg}
                    \label{fig:mesh1}
                \end{figure}
                \begin{equation}
                    \begin{aligned}
                        if\hspace{0.2cm} b^{2}-4ky = 0\\
                        \textbf{degenerate source}
                    \end{aligned}
                \end{equation}
                \newpage
                \begin{figure}[h]
                    \centering
                    \includegraphics[width=0.40\textwidth]{ej5imgb3.jpg}
                    \label{fig:mesh1}
                \end{figure}
            \item How do your results relate to the standard notions of overdamped, critically damped, and underdamped vibrations?\\
                source \xrightarrow{} overdamped\\
                degenerate source \xrightarrow{} critically damped\\
                spiral source \xrightarrow{} underdamped 
            
        \end{enumerate}
        
        
    
    \item (Out of touch with their own feelings) Suppose Romeo and Juliet react to each other, but not to themselves: R˙ = aJ, J˙ = bR. What happens? Analyze the cualitative behaviour depending on the parameters.
        \begin{equation}
            \begin{bmatrix}
                \dot{R}\\
                \dot{J}
            \end{bmatrix}
            = 
            \begin{bmatrix}
                0 & a\\
                b &0
            \end{bmatrix}
            \begin{bmatrix}
                R\\
                J
            \end{bmatrix}
        \end{equation}
        \begin{equation}
            \begin{aligned}
                \Delta= -ba\\
                \tau=0\\
                p(\lambda)= \lambda^{2}-ba=0\\
                \lambda= \pm \sqrt{ba}\\
                if \hspace{0.2cm} ba<0 \xrightarrow{} center\\
                if \hspace{0.2cm} ba>0 \xrightarrow{} \text{saddle node}
            \end{aligned}
        \end{equation}
    







\end{enumerate}
\end{document}
